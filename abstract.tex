\begin{abstract}[Abstract]

The easiest exoplanets to detect are those that orbit very close to their host
stars. As a result, even though these planets are quite rare, they represent a
major fraction of the current exoplanet population. A side-effect of the
proximity between the planet and the star is that the two have strong mutual
interactions through a number of physical processes. One of the most important
of these processes is tides. Tides are thought to shape the orbits of close-in
exoplanets, heat the planet making its radius expand, and even drive some
planets to spiral into their host stars. This chapter briefly introduces the
basics of tidal physics and describes the various fingerprints tides leave
within the observed exoplanet population.

\end{abstract}
