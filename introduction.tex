\section{Introduction}
%
\label{sec:introduction}

The term tides refers to changes in the shape of an astronomical object (in our
case a planet or a star) in response to differences in the gravitational pull of
a nearby mass at different locations within the object. Below, tides and their
effects are discussed qualitatively. For a full mathematical description of
tides see \cite{Murray_Dermott_book}.

Consider an exoplanet system consisting of a single planet orbiting a single
star in a circular orbit. The gravitational acceleration due to the star at the
location of the planet's center provides the centripetal acceleration needed to
keep the planet in orbit. The part of the planet facing the star is closer to
the star and therefore experiences a slightly stronger gravitational pull.
However, if we ignore the rotation of the planet for a moment, that part of the
planet must follow a path with the same radius and period as the center of the
planet (though with its center offset slightly). The result is that that the
star provides larger gravitational acceleration to that part of the planet than
the required centrifugal acceleration. This excess force as referred to as the
tidal force, and it is directed towards the star for the star-facing part of the
planet. Similarly, on the far side of the planet, the star's gravity is slightly
smaller than what is required to keep that part of the planet in orbit,
resulting in a tidal force pointing away from the star. These tidal forces cause
the planet to elongate along the star-planet line, and squeeze in the
perpendicular direction. This elongation is frequently referred to as the tidal
bulge. The amplitude of the tidal bulge is set by competition between the tidal
force and the self-gravity of the planet. Consequently, it is positively
correlated with the planet radius and stellar mass, and negatively correlated
with the planet mass and distance from the star. 

Let us now consider the rotation of the planet. If the period of rotation is
exactly equal to the orbital period (a.k.a. synchronous rotation), the
sub-stellar point is fixed on the planet's surface, and consequently the tidal
bulge is also fixed relative to the planet. However, if the rotation and orbital
periods differ, the tidal bulge will travel on the planet's surface. If the
rotational angular velocity of the planet is $\Omega_{pl}$, and the orbital
angular velocity is $\Omega_{orb}$, a point on the surface of the planet will
travel at a rate of $\Omega_{orb} - \Omega_{pl}$ relative to the sub-stellar
point. Since there are two tidal bulges on the planet (one on the side facing
the star and one on the opposite side), the planetary material will experience a
tidal wave with a frequency of $2(\Omega_{orb} - \Omega_{pl})$.

Any time dependent deformation of the planet will be subject to some amount of
energy dissipation. The exact physical processes leading to this dissipation and
the amount of energy dissipation they drive depend on the internal structure and
properties of the planet. Regardless of the causes, this energy dissipation will
introduce a delay between the tidal forcing and the response of the planet to
that forcing. If the planet spins faster than synchronous (i.e. $\Omega_{pl} >
\Omega_{orb}$), the tidal bulge will be ahead of the sub-stellar point, and if
$\Omega_{pl} < \Omega_{orb}$, the tidal bulge will lag behind. The two tidal
bulges, now shifter relative to the star-planet line, will experience the
gravitational pull of the star. If the bulge is carried ahead of the sub-stellar
point by rotation, the gravitational pull of the star will apply a torque to
the planet opposite to its rotation, acting to slow down its spin, the reaction
force on the star will act to add angular momentum to the orbit. Conversely,
if the tidal bulge lags behind the sub-stellar point, the gravitational pull of
the star will act to spin the planet up, taking angular momentum out of the
orbit.

Everything we have said so far about the tides the star raises on the planet
applies equally to the tides the planet raises on the star. If the spin angular
velocity of the star exceeds the orbital angular velocity, angular momentum is
transferred from the stellar spin to the orbit, and if the stellar spin is
slower than the orbit, angular momentum flows in the opposite direction.
