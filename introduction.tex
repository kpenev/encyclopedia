\section{Introduction}
%
\label{sec:introduction}

The term tides refers to changes in the shape of an astronomical object (in our
case a planet or a star) in response to differences in the gravitational pull of
a nearby mass at different locations within the object. 

Consider an exoplanet system consisting of a single planet orbiting a single
star in a circular orbit. The gravitational acceleration due to the star at the
location of the planet's center provides the centripetal acceleration needed to
keep the planet in orbit. The part of the planet facing the star is closer to
the star and therefore experiences a slightly stronger gravitational pull.
However, if we ignore the rotation of the planet for a moment, that part of the
planet must follow a path with the same radius and period as the center of the
planet (though with its center offset slightly). The result is that that the
star provides larger gravitational acceleration to that part of the planet than
the required centrifugal acceleration. This excess force as referred to as the
tidal force, and it is directed towards the star for the star-facing part of the
planet. Similarly, on the far side of the planet, the star's gravity is slightly
smaller than what is required to keep that part of the planet in orbit,
resulting in a tidal force pointing away from the star. These tidal forces cause
the planet to elongate along the star-planet line, and squeeze in the
perpendicular direction. This elongation is frequently referred to as the tidal
bulge. 
