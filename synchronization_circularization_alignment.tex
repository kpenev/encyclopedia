\section{Tidal circularization, synchronization, and alignment}
%
\label{sec:synchronization_circularization_alignment}

The effect of tides scales as the difference in the gravitational acceleration
due to the companion at the near and far tidal bulges multiplied by the mass in
each bulge. Both of these factors themselves are a strong function of the ratio
of the size of the object to the size of the orbit. As a result, the tidal
coupling between a planet and its host star decreases very rapidly as the
distance between the planet and the star increases. This is why tides are only
important for planets which get very close to their parent stars at least for
some part of their orbit.

As described in the previous section, planetary tides exchange angular
momentum between the orbit and spin of the planet, stellar tides exchange
angular momentum between the orbit and spin of the star, and both tides extract
energy out of the system, leading to orbital circularization. 

The quantity that is most readily affected by tides is the spin of the planet.
This is due to two reasons. First, the angular momentum of the planet is many
orders of magnitude smaller than both the orbital and stellar angular momenta,
so it most easily affect. Second, the self-gravity of the planet is much smaller
than that of the star, and the tidal force on the planet is much larger than
that on the star. Since the size of the tidal bulges is determined by the
competitionbetwen self-gravity and tidal force, the planetary tides are much
stronger. As a consequence, the expectation is that pseudo-synchronizing the
planet's spin with the orbit should be the tidal signature affecting the largest
number of exoplanet systems. Unfortunately, at present, there is no way to
observationally determine the spin of an exoplanet, so we cannot test this
prediction.
