\begin{glossary}[Glossary]

    \term{Hot Jupiter:} an extrasolar planet composed primarily of hydrogen and
    helium (similar to Jupiter and Saturn in our own Solar System) residing very
    close to its parent star.

    \term{Tidal bulge:} stretching of an astrophysical object (planet or star),
    approximately in the direction of a nearby massive companion.

    \term{Tidal dissipation:} heat generated as a tidal bulge moves through a
    planet or star.

    \term{Tidal lag:} offset between the tidal bulge and the line connecting the
    centers of the two objects. Can be defined as either phase lag (the angle
    between the two lines) or time lag (the time between the companion
        culminating from a given point on the surface to the peak of the tidal
    bulge arriving at that point).

    \term{Synchronous rotation:} a state of a planet or star in which the
    object's rotational period is equal to the orbital period.

\end{glossary}
