\begin{glossary}[Key Points]

    \term{Hot Jupiter:} is an extrasolar planet composed primarily of hydrogen
    and helium (similar to Jupiter and Saturn in our own Solar System) residing
    very close to its parent star.

    \term{Tidal bulge:} is the stretching of an astrophysical object (planet or
    star), approximately in the direction of a nearby massive companion.

    \term{Tidal dissipation:} is the conversion of mechanical energy to heat as
    a tidal bulge moves through a planet or star.

    \term{Tidal lag:} is the offset between the tidal bulge and the line
    connecting the centers of the two objects produced by the tidal dissipation.

    \term{Tidal evolution:} is the change in the orbit and spins of the objects
    in an exoplanet system due to the gravitational coupling between the tidal
    bulges on each object with the other object in the system.

    \term{Synchronous rotation:} is a state of a planet or star in which the
    object's rotational period is equal to the orbital period.

    \term{Pseudo-synchronous rotation:} is the rotation of a planet or star with
    such a period that the orbit averaged tidal torque is zero.

    \term{Tidal circularization:} is one of the effects of tidal dissipation
    causing the orbit of a planet to become more and more circular over time.

    \term{Tidal inspiral:} is the shrinking of the orbit of a planet due to
    tides, bringing it closer and closer to its parent star.

    \term{Tidal alignment:} is a process which gradually aligns the spin angular
    momenta of the star and the planet to the orbital angular momentum.

    \term{Tidal heating:} refers to the heat deposited within a planet by tidal
    dissipation. This heat can be a non-negligible contribution to the energy
    budget of the planet and can drive various processes.

\end{glossary}
